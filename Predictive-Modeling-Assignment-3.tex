% Options for packages loaded elsewhere
\PassOptionsToPackage{unicode}{hyperref}
\PassOptionsToPackage{hyphens}{url}
%
\documentclass[
]{article}
\usepackage{amsmath,amssymb}
\usepackage{lmodern}
\usepackage{ifxetex,ifluatex}
\ifnum 0\ifxetex 1\fi\ifluatex 1\fi=0 % if pdftex
  \usepackage[T1]{fontenc}
  \usepackage[utf8]{inputenc}
  \usepackage{textcomp} % provide euro and other symbols
\else % if luatex or xetex
  \usepackage{unicode-math}
  \defaultfontfeatures{Scale=MatchLowercase}
  \defaultfontfeatures[\rmfamily]{Ligatures=TeX,Scale=1}
\fi
% Use upquote if available, for straight quotes in verbatim environments
\IfFileExists{upquote.sty}{\usepackage{upquote}}{}
\IfFileExists{microtype.sty}{% use microtype if available
  \usepackage[]{microtype}
  \UseMicrotypeSet[protrusion]{basicmath} % disable protrusion for tt fonts
}{}
\makeatletter
\@ifundefined{KOMAClassName}{% if non-KOMA class
  \IfFileExists{parskip.sty}{%
    \usepackage{parskip}
  }{% else
    \setlength{\parindent}{0pt}
    \setlength{\parskip}{6pt plus 2pt minus 1pt}}
}{% if KOMA class
  \KOMAoptions{parskip=half}}
\makeatother
\usepackage{xcolor}
\IfFileExists{xurl.sty}{\usepackage{xurl}}{} % add URL line breaks if available
\IfFileExists{bookmark.sty}{\usepackage{bookmark}}{\usepackage{hyperref}}
\hypersetup{
  pdftitle={Assignment 3},
  pdfauthor={Temi Ajulo},
  hidelinks,
  pdfcreator={LaTeX via pandoc}}
\urlstyle{same} % disable monospaced font for URLs
\usepackage[margin=1in]{geometry}
\usepackage{color}
\usepackage{fancyvrb}
\newcommand{\VerbBar}{|}
\newcommand{\VERB}{\Verb[commandchars=\\\{\}]}
\DefineVerbatimEnvironment{Highlighting}{Verbatim}{commandchars=\\\{\}}
% Add ',fontsize=\small' for more characters per line
\usepackage{framed}
\definecolor{shadecolor}{RGB}{248,248,248}
\newenvironment{Shaded}{\begin{snugshade}}{\end{snugshade}}
\newcommand{\AlertTok}[1]{\textcolor[rgb]{0.94,0.16,0.16}{#1}}
\newcommand{\AnnotationTok}[1]{\textcolor[rgb]{0.56,0.35,0.01}{\textbf{\textit{#1}}}}
\newcommand{\AttributeTok}[1]{\textcolor[rgb]{0.77,0.63,0.00}{#1}}
\newcommand{\BaseNTok}[1]{\textcolor[rgb]{0.00,0.00,0.81}{#1}}
\newcommand{\BuiltInTok}[1]{#1}
\newcommand{\CharTok}[1]{\textcolor[rgb]{0.31,0.60,0.02}{#1}}
\newcommand{\CommentTok}[1]{\textcolor[rgb]{0.56,0.35,0.01}{\textit{#1}}}
\newcommand{\CommentVarTok}[1]{\textcolor[rgb]{0.56,0.35,0.01}{\textbf{\textit{#1}}}}
\newcommand{\ConstantTok}[1]{\textcolor[rgb]{0.00,0.00,0.00}{#1}}
\newcommand{\ControlFlowTok}[1]{\textcolor[rgb]{0.13,0.29,0.53}{\textbf{#1}}}
\newcommand{\DataTypeTok}[1]{\textcolor[rgb]{0.13,0.29,0.53}{#1}}
\newcommand{\DecValTok}[1]{\textcolor[rgb]{0.00,0.00,0.81}{#1}}
\newcommand{\DocumentationTok}[1]{\textcolor[rgb]{0.56,0.35,0.01}{\textbf{\textit{#1}}}}
\newcommand{\ErrorTok}[1]{\textcolor[rgb]{0.64,0.00,0.00}{\textbf{#1}}}
\newcommand{\ExtensionTok}[1]{#1}
\newcommand{\FloatTok}[1]{\textcolor[rgb]{0.00,0.00,0.81}{#1}}
\newcommand{\FunctionTok}[1]{\textcolor[rgb]{0.00,0.00,0.00}{#1}}
\newcommand{\ImportTok}[1]{#1}
\newcommand{\InformationTok}[1]{\textcolor[rgb]{0.56,0.35,0.01}{\textbf{\textit{#1}}}}
\newcommand{\KeywordTok}[1]{\textcolor[rgb]{0.13,0.29,0.53}{\textbf{#1}}}
\newcommand{\NormalTok}[1]{#1}
\newcommand{\OperatorTok}[1]{\textcolor[rgb]{0.81,0.36,0.00}{\textbf{#1}}}
\newcommand{\OtherTok}[1]{\textcolor[rgb]{0.56,0.35,0.01}{#1}}
\newcommand{\PreprocessorTok}[1]{\textcolor[rgb]{0.56,0.35,0.01}{\textit{#1}}}
\newcommand{\RegionMarkerTok}[1]{#1}
\newcommand{\SpecialCharTok}[1]{\textcolor[rgb]{0.00,0.00,0.00}{#1}}
\newcommand{\SpecialStringTok}[1]{\textcolor[rgb]{0.31,0.60,0.02}{#1}}
\newcommand{\StringTok}[1]{\textcolor[rgb]{0.31,0.60,0.02}{#1}}
\newcommand{\VariableTok}[1]{\textcolor[rgb]{0.00,0.00,0.00}{#1}}
\newcommand{\VerbatimStringTok}[1]{\textcolor[rgb]{0.31,0.60,0.02}{#1}}
\newcommand{\WarningTok}[1]{\textcolor[rgb]{0.56,0.35,0.01}{\textbf{\textit{#1}}}}
\usepackage{graphicx}
\makeatletter
\def\maxwidth{\ifdim\Gin@nat@width>\linewidth\linewidth\else\Gin@nat@width\fi}
\def\maxheight{\ifdim\Gin@nat@height>\textheight\textheight\else\Gin@nat@height\fi}
\makeatother
% Scale images if necessary, so that they will not overflow the page
% margins by default, and it is still possible to overwrite the defaults
% using explicit options in \includegraphics[width, height, ...]{}
\setkeys{Gin}{width=\maxwidth,height=\maxheight,keepaspectratio}
% Set default figure placement to htbp
\makeatletter
\def\fps@figure{htbp}
\makeatother
\setlength{\emergencystretch}{3em} % prevent overfull lines
\providecommand{\tightlist}{%
  \setlength{\itemsep}{0pt}\setlength{\parskip}{0pt}}
\setcounter{secnumdepth}{-\maxdimen} % remove section numbering
\ifluatex
  \usepackage{selnolig}  % disable illegal ligatures
\fi

\title{Assignment 3}
\author{Temi Ajulo}
\date{11/14/2021}

\begin{document}
\maketitle

\begin{enumerate}
\def\labelenumi{\arabic{enumi}.}
\setcounter{enumi}{9}
\tightlist
\item
  This question should be answered using the Weekly data set, which is
  part of the ISLR package. This data is similar in nature to the
  Smarket data from this chapter's lab, except that it contains 1, 089
  weekly returns for 21 years, from the beginning of 1990 to the end of
  2010
\end{enumerate}

\begin{Shaded}
\begin{Highlighting}[]
\FunctionTok{library}\NormalTok{(ISLR)}
\end{Highlighting}
\end{Shaded}

\begin{verbatim}
## Warning: package 'ISLR' was built under R version 4.1.1
\end{verbatim}

\begin{Shaded}
\begin{Highlighting}[]
\FunctionTok{require}\NormalTok{(MASS)}
\end{Highlighting}
\end{Shaded}

\begin{verbatim}
## Loading required package: MASS
\end{verbatim}

\begin{Shaded}
\begin{Highlighting}[]
\FunctionTok{require}\NormalTok{(class)}
\end{Highlighting}
\end{Shaded}

\begin{verbatim}
## Loading required package: class
\end{verbatim}

\begin{enumerate}
\def\labelenumi{(\alph{enumi})}
\tightlist
\item
  Produce some numerical and graphical summaries of the Weekly data. Do
  there appear to be any patterns?
\end{enumerate}

\begin{Shaded}
\begin{Highlighting}[]
\FunctionTok{summary}\NormalTok{(Weekly)}
\end{Highlighting}
\end{Shaded}

\begin{verbatim}
##       Year           Lag1               Lag2               Lag3         
##  Min.   :1990   Min.   :-18.1950   Min.   :-18.1950   Min.   :-18.1950  
##  1st Qu.:1995   1st Qu.: -1.1540   1st Qu.: -1.1540   1st Qu.: -1.1580  
##  Median :2000   Median :  0.2410   Median :  0.2410   Median :  0.2410  
##  Mean   :2000   Mean   :  0.1506   Mean   :  0.1511   Mean   :  0.1472  
##  3rd Qu.:2005   3rd Qu.:  1.4050   3rd Qu.:  1.4090   3rd Qu.:  1.4090  
##  Max.   :2010   Max.   : 12.0260   Max.   : 12.0260   Max.   : 12.0260  
##       Lag4               Lag5              Volume            Today         
##  Min.   :-18.1950   Min.   :-18.1950   Min.   :0.08747   Min.   :-18.1950  
##  1st Qu.: -1.1580   1st Qu.: -1.1660   1st Qu.:0.33202   1st Qu.: -1.1540  
##  Median :  0.2380   Median :  0.2340   Median :1.00268   Median :  0.2410  
##  Mean   :  0.1458   Mean   :  0.1399   Mean   :1.57462   Mean   :  0.1499  
##  3rd Qu.:  1.4090   3rd Qu.:  1.4050   3rd Qu.:2.05373   3rd Qu.:  1.4050  
##  Max.   : 12.0260   Max.   : 12.0260   Max.   :9.32821   Max.   : 12.0260  
##  Direction 
##  Down:484  
##  Up  :605  
##            
##            
##            
## 
\end{verbatim}

\begin{Shaded}
\begin{Highlighting}[]
\FunctionTok{plot}\NormalTok{(Today}\SpecialCharTok{\textasciitilde{}}\NormalTok{Lag1, }\AttributeTok{col=}\StringTok{"darkred"}\NormalTok{, }\AttributeTok{data=}\NormalTok{Weekly)}
\NormalTok{simplelm }\OtherTok{=} \FunctionTok{lm}\NormalTok{(Today}\SpecialCharTok{\textasciitilde{}}\NormalTok{Lag1, }\AttributeTok{data=}\NormalTok{Weekly)}
\FunctionTok{abline}\NormalTok{(simplelm, }\AttributeTok{lwd=} \DecValTok{3}\NormalTok{, }\AttributeTok{col=} \StringTok{"darkblue"}\NormalTok{)}
\end{Highlighting}
\end{Shaded}

\includegraphics{Predictive-Modeling-Assignment-3_files/figure-latex/unnamed-chunk-2-1.pdf}

\begin{Shaded}
\begin{Highlighting}[]
\FunctionTok{pairs}\NormalTok{(Weekly)}
\end{Highlighting}
\end{Shaded}

\includegraphics{Predictive-Modeling-Assignment-3_files/figure-latex/unnamed-chunk-2-2.pdf}
(b) Use the full data set to perform a logistic regression with
Direction as the response and the five lag variables plus Volume as
predictors. Use the summary function to print the results. Do any of the
predictors appear to be statistically significant? If so,which ones?

\begin{Shaded}
\begin{Highlighting}[]
\NormalTok{logmod }\OtherTok{=} \FunctionTok{glm}\NormalTok{(Direction}\SpecialCharTok{\textasciitilde{}}\NormalTok{Lag1}\SpecialCharTok{+}\NormalTok{Lag2}\SpecialCharTok{+}\NormalTok{Lag3}\SpecialCharTok{+}\NormalTok{Lag4}\SpecialCharTok{+}\NormalTok{Lag5}\SpecialCharTok{+}\NormalTok{Volume,}\AttributeTok{family =} \StringTok{"binomial"}\NormalTok{, }\AttributeTok{data=}\NormalTok{Weekly)}
\FunctionTok{summary}\NormalTok{(logmod)}
\end{Highlighting}
\end{Shaded}

\begin{verbatim}
## 
## Call:
## glm(formula = Direction ~ Lag1 + Lag2 + Lag3 + Lag4 + Lag5 + 
##     Volume, family = "binomial", data = Weekly)
## 
## Deviance Residuals: 
##     Min       1Q   Median       3Q      Max  
## -1.6949  -1.2565   0.9913   1.0849   1.4579  
## 
## Coefficients:
##             Estimate Std. Error z value Pr(>|z|)   
## (Intercept)  0.26686    0.08593   3.106   0.0019 **
## Lag1        -0.04127    0.02641  -1.563   0.1181   
## Lag2         0.05844    0.02686   2.175   0.0296 * 
## Lag3        -0.01606    0.02666  -0.602   0.5469   
## Lag4        -0.02779    0.02646  -1.050   0.2937   
## Lag5        -0.01447    0.02638  -0.549   0.5833   
## Volume      -0.02274    0.03690  -0.616   0.5377   
## ---
## Signif. codes:  0 '***' 0.001 '**' 0.01 '*' 0.05 '.' 0.1 ' ' 1
## 
## (Dispersion parameter for binomial family taken to be 1)
## 
##     Null deviance: 1496.2  on 1088  degrees of freedom
## Residual deviance: 1486.4  on 1082  degrees of freedom
## AIC: 1500.4
## 
## Number of Fisher Scoring iterations: 4
\end{verbatim}

Ans:Out of all six predictors, only lag2 shows to be statistically
significant to predict the direction. The estimated coefficient is
0.058, which would mean that an increase in Lag2 represents an increase
of e\^{}0.058=1,06 in the odds of direction going up.

\begin{enumerate}
\def\labelenumi{(\alph{enumi})}
\setcounter{enumi}{2}
\tightlist
\item
  Compute the confusion matrix and overall fraction of correct
  predictions. Explain what the confusion matrix is telling you about
  the types of mistakes made by logistic regression
\end{enumerate}

\begin{Shaded}
\begin{Highlighting}[]
\NormalTok{probs }\OtherTok{=} \FunctionTok{predict}\NormalTok{(logmod, }\AttributeTok{type=}\StringTok{"response"}\NormalTok{)}
\NormalTok{preds }\OtherTok{=} \FunctionTok{rep}\NormalTok{(}\StringTok{"Down"}\NormalTok{, }\DecValTok{1089}\NormalTok{)}
\NormalTok{preds[probs }\SpecialCharTok{\textgreater{}} \FloatTok{0.5}\NormalTok{] }\OtherTok{=} \StringTok{"Up"}
\FunctionTok{table}\NormalTok{(preds, Weekly}\SpecialCharTok{$}\NormalTok{Direction)}
\end{Highlighting}
\end{Shaded}

\begin{verbatim}
##       
## preds  Down  Up
##   Down   54  48
##   Up    430 557
\end{verbatim}

\begin{Shaded}
\begin{Highlighting}[]
\FunctionTok{plot}\NormalTok{(probs, }\AttributeTok{col=} \FunctionTok{ifelse}\NormalTok{(Weekly}\SpecialCharTok{$}\NormalTok{Direction}\SpecialCharTok{==}\StringTok{"Down"}\NormalTok{, }\StringTok{"blue"}\NormalTok{,}\StringTok{"green"}\NormalTok{), }\AttributeTok{pch=}\DecValTok{16}\NormalTok{)}
\FunctionTok{abline}\NormalTok{(}\AttributeTok{h =} \FloatTok{0.5}\NormalTok{, }\AttributeTok{lwd=} \DecValTok{3}\NormalTok{)}
\end{Highlighting}
\end{Shaded}

\includegraphics{Predictive-Modeling-Assignment-3_files/figure-latex/unnamed-chunk-4-1.pdf}

\begin{Shaded}
\begin{Highlighting}[]
\FunctionTok{hist}\NormalTok{(probs, }\AttributeTok{breaks=} \DecValTok{100}\NormalTok{, }\AttributeTok{col=} \StringTok{"darkorange"}\NormalTok{)}
\FunctionTok{abline}\NormalTok{(}\AttributeTok{v =} \FunctionTok{mean}\NormalTok{(probs), }\AttributeTok{lwd =} \DecValTok{2}\NormalTok{)}
\end{Highlighting}
\end{Shaded}

\includegraphics{Predictive-Modeling-Assignment-3_files/figure-latex/unnamed-chunk-4-2.pdf}
Ans:Based on the confusion matrix results, we can see that we are not
conservative to consider the results as going up, and as such, we
predict most of the cases to up go UP. Of course, this means that we
have found a high percentage of true positives, but with an ample tread
off, and that is that we also find 430 False Positives. We only consider
a negative 54/484 of the true negatives, a low proportion. Ans:In the
index plot we can see as most of our probabilities of being ``Up'' are
over 0.5. So, if we set the threshold to consider one sample as going Up
as 0.5,

\begin{enumerate}
\def\labelenumi{(\alph{enumi})}
\setcounter{enumi}{3}
\tightlist
\item
  Now fit the logistic regression model using a training data period
  from 1990 to 2008, with Lag2 as the only predictor. Compute the
  confusion matrix and the overall fraction of correct predictions for
  the held out data (that is, the data from 2009 and 2010).
\end{enumerate}

\begin{Shaded}
\begin{Highlighting}[]
\NormalTok{training.data }\OtherTok{=}\NormalTok{ Weekly[Weekly}\SpecialCharTok{$}\NormalTok{Year}\SpecialCharTok{\textless{}}\DecValTok{2009}\NormalTok{,]}
\NormalTok{test.data }\OtherTok{=}\NormalTok{ Weekly[Weekly}\SpecialCharTok{$}\NormalTok{Year}\SpecialCharTok{\textgreater{}}\DecValTok{2008}\NormalTok{,]}
\NormalTok{simpglm }\OtherTok{=} \FunctionTok{glm}\NormalTok{(Direction}\SpecialCharTok{\textasciitilde{}}\NormalTok{Lag2, }\AttributeTok{data=}\NormalTok{ training.data, }\AttributeTok{family =} \StringTok{"binomial"}\NormalTok{)}
\FunctionTok{summary}\NormalTok{(simpglm)}
\end{Highlighting}
\end{Shaded}

\begin{verbatim}
## 
## Call:
## glm(formula = Direction ~ Lag2, family = "binomial", data = training.data)
## 
## Deviance Residuals: 
##    Min      1Q  Median      3Q     Max  
## -1.536  -1.264   1.021   1.091   1.368  
## 
## Coefficients:
##             Estimate Std. Error z value Pr(>|z|)   
## (Intercept)  0.20326    0.06428   3.162  0.00157 **
## Lag2         0.05810    0.02870   2.024  0.04298 * 
## ---
## Signif. codes:  0 '***' 0.001 '**' 0.01 '*' 0.05 '.' 0.1 ' ' 1
## 
## (Dispersion parameter for binomial family taken to be 1)
## 
##     Null deviance: 1354.7  on 984  degrees of freedom
## Residual deviance: 1350.5  on 983  degrees of freedom
## AIC: 1354.5
## 
## Number of Fisher Scoring iterations: 4
\end{verbatim}

\begin{Shaded}
\begin{Highlighting}[]
\NormalTok{testprobs }\OtherTok{=} \FunctionTok{predict}\NormalTok{(simpglm, }\AttributeTok{type=}\StringTok{"response"}\NormalTok{, }\AttributeTok{newdata =}\NormalTok{ test.data)}
\NormalTok{testdirs }\OtherTok{=}\NormalTok{ Weekly}\SpecialCharTok{$}\NormalTok{Direction[Weekly}\SpecialCharTok{$}\NormalTok{Year}\SpecialCharTok{\textgreater{}}\DecValTok{2008}\NormalTok{]}
\FunctionTok{plot}\NormalTok{(testprobs, }\AttributeTok{col=} \FunctionTok{ifelse}\NormalTok{(Weekly}\SpecialCharTok{$}\NormalTok{Direction[Weekly}\SpecialCharTok{$}\NormalTok{Year}\SpecialCharTok{\textgreater{}}\DecValTok{2008}\NormalTok{]}\SpecialCharTok{==}\StringTok{"Down"}\NormalTok{, }\StringTok{"blue"}\NormalTok{,}\StringTok{"green"}\NormalTok{), }\AttributeTok{pch=}\DecValTok{16}\NormalTok{)}
\FunctionTok{abline}\NormalTok{(}\AttributeTok{h =} \FloatTok{0.5}\NormalTok{, }\AttributeTok{lwd=} \DecValTok{3}\NormalTok{)}
\end{Highlighting}
\end{Shaded}

\includegraphics{Predictive-Modeling-Assignment-3_files/figure-latex/unnamed-chunk-5-1.pdf}

\begin{Shaded}
\begin{Highlighting}[]
\NormalTok{testpreds }\OtherTok{=} \FunctionTok{rep}\NormalTok{(}\StringTok{"Down"}\NormalTok{, }\DecValTok{104}\NormalTok{)}
\NormalTok{testpreds[testprobs}\SpecialCharTok{\textgreater{}}\FloatTok{0.5}\NormalTok{] }\OtherTok{=} \StringTok{"Up"}
\FunctionTok{mean}\NormalTok{(probs)}
\end{Highlighting}
\end{Shaded}

\begin{verbatim}
## [1] 0.5555556
\end{verbatim}

\begin{Shaded}
\begin{Highlighting}[]
\FunctionTok{table}\NormalTok{(testpreds, testdirs)}
\end{Highlighting}
\end{Shaded}

\begin{verbatim}
##          testdirs
## testpreds Down Up
##      Down    9  5
##      Up     34 56
\end{verbatim}

Ans:The test error rate for the logistic regression is 37.5\%

\begin{enumerate}
\def\labelenumi{(\alph{enumi})}
\setcounter{enumi}{4}
\tightlist
\item
  Repeat (d) using LDA
\end{enumerate}

\begin{Shaded}
\begin{Highlighting}[]
\NormalTok{lda.fit }\OtherTok{=} \FunctionTok{lda}\NormalTok{(Direction}\SpecialCharTok{\textasciitilde{}}\NormalTok{Lag2, }\AttributeTok{data=}\NormalTok{ training.data)}
\NormalTok{lda.fit}
\end{Highlighting}
\end{Shaded}

\begin{verbatim}
## Call:
## lda(Direction ~ Lag2, data = training.data)
## 
## Prior probabilities of groups:
##      Down        Up 
## 0.4477157 0.5522843 
## 
## Group means:
##             Lag2
## Down -0.03568254
## Up    0.26036581
## 
## Coefficients of linear discriminants:
##            LD1
## Lag2 0.4414162
\end{verbatim}

\begin{Shaded}
\begin{Highlighting}[]
\FunctionTok{par}\NormalTok{(}\AttributeTok{mar=}\FunctionTok{c}\NormalTok{(}\DecValTok{2}\NormalTok{,}\DecValTok{2}\NormalTok{,}\DecValTok{2}\NormalTok{,}\DecValTok{2}\NormalTok{))}
\FunctionTok{plot}\NormalTok{(lda.fit)}
\end{Highlighting}
\end{Shaded}

\includegraphics{Predictive-Modeling-Assignment-3_files/figure-latex/unnamed-chunk-6-1.pdf}

\begin{Shaded}
\begin{Highlighting}[]
\NormalTok{lda.pred }\OtherTok{=} \FunctionTok{predict}\NormalTok{(lda.fit, }\AttributeTok{newdata=}\NormalTok{test.data, }\AttributeTok{type=}\StringTok{"response"}\NormalTok{)}
\NormalTok{lda.class }\OtherTok{=}\NormalTok{ lda.pred}\SpecialCharTok{$}\NormalTok{class}
\FunctionTok{table}\NormalTok{(lda.class, test.data}\SpecialCharTok{$}\NormalTok{Direction)}
\end{Highlighting}
\end{Shaded}

\begin{verbatim}
##          
## lda.class Down Up
##      Down    9  5
##      Up     34 56
\end{verbatim}

Ans:test error is 37.7\%

\begin{enumerate}
\def\labelenumi{(\alph{enumi})}
\setcounter{enumi}{5}
\tightlist
\item
  Repeat (d) using QDA
\end{enumerate}

\begin{Shaded}
\begin{Highlighting}[]
\NormalTok{qda.fit }\OtherTok{=} \FunctionTok{qda}\NormalTok{(Direction}\SpecialCharTok{\textasciitilde{}}\NormalTok{Lag2, }\AttributeTok{data=}\NormalTok{ training.data)}
\NormalTok{qda.fit}
\end{Highlighting}
\end{Shaded}

\begin{verbatim}
## Call:
## qda(Direction ~ Lag2, data = training.data)
## 
## Prior probabilities of groups:
##      Down        Up 
## 0.4477157 0.5522843 
## 
## Group means:
##             Lag2
## Down -0.03568254
## Up    0.26036581
\end{verbatim}

\begin{Shaded}
\begin{Highlighting}[]
\NormalTok{qda.pred }\OtherTok{=} \FunctionTok{predict}\NormalTok{(qda.fit, }\AttributeTok{newdata=}\NormalTok{test.data, }\AttributeTok{type=}\StringTok{"response"}\NormalTok{)}
\NormalTok{qda.class }\OtherTok{=}\NormalTok{ qda.pred}\SpecialCharTok{$}\NormalTok{class}
\FunctionTok{table}\NormalTok{(qda.class, test.data}\SpecialCharTok{$}\NormalTok{Direction)}
\end{Highlighting}
\end{Shaded}

\begin{verbatim}
##          
## qda.class Down Up
##      Down    0  0
##      Up     43 61
\end{verbatim}

Ans:test error is 41,35\%

\begin{enumerate}
\def\labelenumi{(\alph{enumi})}
\setcounter{enumi}{6}
\tightlist
\item
  Repeat (d) using KNN with K = 1.
\end{enumerate}

\begin{Shaded}
\begin{Highlighting}[]
\FunctionTok{set.seed}\NormalTok{(}\DecValTok{1}\NormalTok{)}
\NormalTok{train.X }\OtherTok{=} \FunctionTok{cbind}\NormalTok{(training.data}\SpecialCharTok{$}\NormalTok{Lag2)}
\NormalTok{test.X }\OtherTok{=} \FunctionTok{cbind}\NormalTok{(test.data}\SpecialCharTok{$}\NormalTok{Lag2)}
\NormalTok{train.Y }\OtherTok{=} \FunctionTok{cbind}\NormalTok{(training.data}\SpecialCharTok{$}\NormalTok{Direction)}
\NormalTok{knn.pred }\OtherTok{=} \FunctionTok{knn}\NormalTok{(train.X, test.X, train.Y, }\AttributeTok{k=}\DecValTok{1}\NormalTok{)}
\FunctionTok{table}\NormalTok{(knn.pred, test.data}\SpecialCharTok{$}\NormalTok{Direction)}
\end{Highlighting}
\end{Shaded}

\begin{verbatim}
##         
## knn.pred Down Up
##        1   21 30
##        2   22 31
\end{verbatim}

\begin{enumerate}
\def\labelenumi{(\alph{enumi})}
\setcounter{enumi}{7}
\tightlist
\item
  Which of these methods appears to provide the best results on this
  data?
\end{enumerate}

Ans:The methods that have the highest accuracy rates are the Logistic
Regression and Linear Discriminant Analysis

\begin{enumerate}
\def\labelenumi{(\roman{enumi})}
\tightlist
\item
  Experiment with different combinations of predictors, including
  possible transformations and interactions, for each of the methods.
  Report the variables, method, and associated confusion matrix that
  appears to provide the best results on the held out data. Note that
  you should also experiment with values forK in the KNN classifier.
\end{enumerate}

\begin{Shaded}
\begin{Highlighting}[]
\NormalTok{qda.fit2 }\OtherTok{=} \FunctionTok{qda}\NormalTok{(Direction}\SpecialCharTok{\textasciitilde{}}\NormalTok{Lag1 }\SpecialCharTok{+}\NormalTok{ Lag2 }\SpecialCharTok{+}\NormalTok{ Lag4, }\AttributeTok{data=}\NormalTok{ training.data)}
\NormalTok{qda.fit2}
\end{Highlighting}
\end{Shaded}

\begin{verbatim}
## Call:
## qda(Direction ~ Lag1 + Lag2 + Lag4, data = training.data)
## 
## Prior probabilities of groups:
##      Down        Up 
## 0.4477157 0.5522843 
## 
## Group means:
##              Lag1        Lag2       Lag4
## Down  0.289444444 -0.03568254 0.15925624
## Up   -0.009213235  0.26036581 0.09220956
\end{verbatim}

\begin{Shaded}
\begin{Highlighting}[]
\NormalTok{qda.pred2 }\OtherTok{=} \FunctionTok{predict}\NormalTok{(qda.fit2, }\AttributeTok{newdata=}\NormalTok{test.data, }\AttributeTok{type=}\StringTok{"response"}\NormalTok{)}
\NormalTok{qda.class2 }\OtherTok{=}\NormalTok{ qda.pred2}\SpecialCharTok{$}\NormalTok{class}
\FunctionTok{table}\NormalTok{(qda.class2, test.data}\SpecialCharTok{$}\NormalTok{Direction)}
\end{Highlighting}
\end{Shaded}

\begin{verbatim}
##           
## qda.class2 Down Up
##       Down    9 20
##       Up     34 41
\end{verbatim}

\begin{Shaded}
\begin{Highlighting}[]
\NormalTok{lda.fit2 }\OtherTok{=} \FunctionTok{lda}\NormalTok{(Direction}\SpecialCharTok{\textasciitilde{}}\NormalTok{Lag1 }\SpecialCharTok{+}\NormalTok{ Lag2 }\SpecialCharTok{+}\NormalTok{ Lag4, }\AttributeTok{data=}\NormalTok{ training.data)}
\NormalTok{lda.fit2}
\end{Highlighting}
\end{Shaded}

\begin{verbatim}
## Call:
## lda(Direction ~ Lag1 + Lag2 + Lag4, data = training.data)
## 
## Prior probabilities of groups:
##      Down        Up 
## 0.4477157 0.5522843 
## 
## Group means:
##              Lag1        Lag2       Lag4
## Down  0.289444444 -0.03568254 0.15925624
## Up   -0.009213235  0.26036581 0.09220956
## 
## Coefficients of linear discriminants:
##             LD1
## Lag1 -0.2984478
## Lag2  0.2960224
## Lag4 -0.1113485
\end{verbatim}

\begin{Shaded}
\begin{Highlighting}[]
\NormalTok{lda.pred2 }\OtherTok{=} \FunctionTok{predict}\NormalTok{(lda.fit2, }\AttributeTok{newdata=}\NormalTok{test.data, }\AttributeTok{type=}\StringTok{"response"}\NormalTok{)}
\NormalTok{lda.class2 }\OtherTok{=}\NormalTok{ lda.pred2}\SpecialCharTok{$}\NormalTok{class}
\FunctionTok{table}\NormalTok{(lda.class2, test.data}\SpecialCharTok{$}\NormalTok{Direction)}
\end{Highlighting}
\end{Shaded}

\begin{verbatim}
##           
## lda.class2 Down Up
##       Down    9  7
##       Up     34 54
\end{verbatim}

\begin{enumerate}
\def\labelenumi{\arabic{enumi}.}
\setcounter{enumi}{12}
\tightlist
\item
  Using the Boston data set, fit classification models in order to
  predict whether a given suburb has a crime rate above or below the
  median.Explore logistic regression, LDA, and KNN models using various
  subsets of the predictors. Describe your findings
\end{enumerate}

\begin{Shaded}
\begin{Highlighting}[]
\FunctionTok{summary}\NormalTok{(Boston)}
\end{Highlighting}
\end{Shaded}

\begin{verbatim}
##       crim                zn             indus            chas        
##  Min.   : 0.00632   Min.   :  0.00   Min.   : 0.46   Min.   :0.00000  
##  1st Qu.: 0.08205   1st Qu.:  0.00   1st Qu.: 5.19   1st Qu.:0.00000  
##  Median : 0.25651   Median :  0.00   Median : 9.69   Median :0.00000  
##  Mean   : 3.61352   Mean   : 11.36   Mean   :11.14   Mean   :0.06917  
##  3rd Qu.: 3.67708   3rd Qu.: 12.50   3rd Qu.:18.10   3rd Qu.:0.00000  
##  Max.   :88.97620   Max.   :100.00   Max.   :27.74   Max.   :1.00000  
##       nox               rm             age              dis        
##  Min.   :0.3850   Min.   :3.561   Min.   :  2.90   Min.   : 1.130  
##  1st Qu.:0.4490   1st Qu.:5.886   1st Qu.: 45.02   1st Qu.: 2.100  
##  Median :0.5380   Median :6.208   Median : 77.50   Median : 3.207  
##  Mean   :0.5547   Mean   :6.285   Mean   : 68.57   Mean   : 3.795  
##  3rd Qu.:0.6240   3rd Qu.:6.623   3rd Qu.: 94.08   3rd Qu.: 5.188  
##  Max.   :0.8710   Max.   :8.780   Max.   :100.00   Max.   :12.127  
##       rad              tax           ptratio          black       
##  Min.   : 1.000   Min.   :187.0   Min.   :12.60   Min.   :  0.32  
##  1st Qu.: 4.000   1st Qu.:279.0   1st Qu.:17.40   1st Qu.:375.38  
##  Median : 5.000   Median :330.0   Median :19.05   Median :391.44  
##  Mean   : 9.549   Mean   :408.2   Mean   :18.46   Mean   :356.67  
##  3rd Qu.:24.000   3rd Qu.:666.0   3rd Qu.:20.20   3rd Qu.:396.23  
##  Max.   :24.000   Max.   :711.0   Max.   :22.00   Max.   :396.90  
##      lstat            medv      
##  Min.   : 1.73   Min.   : 5.00  
##  1st Qu.: 6.95   1st Qu.:17.02  
##  Median :11.36   Median :21.20  
##  Mean   :12.65   Mean   :22.53  
##  3rd Qu.:16.95   3rd Qu.:25.00  
##  Max.   :37.97   Max.   :50.00
\end{verbatim}

\begin{Shaded}
\begin{Highlighting}[]
\FunctionTok{attach}\NormalTok{(Boston)}
\end{Highlighting}
\end{Shaded}

Creating binary crim variable.

\begin{Shaded}
\begin{Highlighting}[]
\NormalTok{crime01 }\OtherTok{\textless{}{-}} \FunctionTok{rep}\NormalTok{(}\DecValTok{0}\NormalTok{, }\FunctionTok{length}\NormalTok{(crim))}
\NormalTok{crime01[crim }\SpecialCharTok{\textgreater{}} \FunctionTok{median}\NormalTok{(crim)] }\OtherTok{\textless{}{-}} \DecValTok{1}
\NormalTok{Boston}\OtherTok{=} \FunctionTok{data.frame}\NormalTok{(Boston,crime01)}
\end{Highlighting}
\end{Shaded}

Splitting the dataset

\begin{Shaded}
\begin{Highlighting}[]
\NormalTok{train }\OtherTok{=} \DecValTok{1}\SpecialCharTok{:}\NormalTok{(}\FunctionTok{dim}\NormalTok{(Boston)[}\DecValTok{1}\NormalTok{]}\SpecialCharTok{/}\DecValTok{2}\NormalTok{)}
\NormalTok{test }\OtherTok{=}\NormalTok{ (}\FunctionTok{dim}\NormalTok{(Boston)[}\DecValTok{1}\NormalTok{]}\SpecialCharTok{/}\DecValTok{2} \SpecialCharTok{+} \DecValTok{1}\NormalTok{)}\SpecialCharTok{:}\FunctionTok{dim}\NormalTok{(Boston)[}\DecValTok{1}\NormalTok{]}
\NormalTok{Boston.train }\OtherTok{=}\NormalTok{ Boston[train, ]}
\NormalTok{Boston.test }\OtherTok{=}\NormalTok{ Boston[test, ]}
\NormalTok{crime01.test }\OtherTok{=}\NormalTok{ crime01[test]}
\end{Highlighting}
\end{Shaded}

Determination of any associations to crime01

\begin{Shaded}
\begin{Highlighting}[]
\FunctionTok{pairs}\NormalTok{(Boston)}
\end{Highlighting}
\end{Shaded}

\includegraphics{Predictive-Modeling-Assignment-3_files/figure-latex/unnamed-chunk-13-1.pdf}
Ans:Looking at the binary correlation graphs, it looks like the
variables nox, age, dis, and medv might correlate with our crime rate.
Therefore, these may be good predictors for our crim\_lvl variable.We
will first run a logistic regression model to determine which of our
predictors are significant, then evaluate how effective our model
predicts whether the crime rate will be above or below the median.

Logistic Regression

\begin{Shaded}
\begin{Highlighting}[]
\FunctionTok{set.seed}\NormalTok{(}\DecValTok{1}\NormalTok{)}
\NormalTok{Boston.fit }\OtherTok{\textless{}{-}}\FunctionTok{glm}\NormalTok{(crime01}\SpecialCharTok{\textasciitilde{}}\NormalTok{ indus}\SpecialCharTok{+}\NormalTok{nox}\SpecialCharTok{+}\NormalTok{age}\SpecialCharTok{+}\NormalTok{dis}\SpecialCharTok{+}\NormalTok{rad}\SpecialCharTok{+}\NormalTok{tax, }\AttributeTok{data=}\NormalTok{Boston.train,}\AttributeTok{family=}\NormalTok{binomial)}
\NormalTok{Boston.probs }\OtherTok{=} \FunctionTok{predict}\NormalTok{(Boston.fit, Boston.test, }\AttributeTok{type =} \StringTok{"response"}\NormalTok{)}
\NormalTok{Boston.pred }\OtherTok{=} \FunctionTok{rep}\NormalTok{(}\DecValTok{0}\NormalTok{, }\FunctionTok{length}\NormalTok{(Boston.probs))}
\NormalTok{Boston.pred[Boston.probs }\SpecialCharTok{\textgreater{}} \FloatTok{0.5}\NormalTok{] }\OtherTok{=} \DecValTok{1}
\FunctionTok{table}\NormalTok{(Boston.pred, crime01.test)}
\end{Highlighting}
\end{Shaded}

\begin{verbatim}
##            crime01.test
## Boston.pred   0   1
##           0  75   8
##           1  15 155
\end{verbatim}

\begin{Shaded}
\begin{Highlighting}[]
\FunctionTok{mean}\NormalTok{(Boston.pred }\SpecialCharTok{!=}\NormalTok{ crime01.test)}
\end{Highlighting}
\end{Shaded}

\begin{verbatim}
## [1] 0.09090909
\end{verbatim}

\begin{Shaded}
\begin{Highlighting}[]
\FunctionTok{summary}\NormalTok{(Boston.fit)}
\end{Highlighting}
\end{Shaded}

\begin{verbatim}
## 
## Call:
## glm(formula = crime01 ~ indus + nox + age + dis + rad + tax, 
##     family = binomial, data = Boston.train)
## 
## Deviance Residuals: 
##      Min        1Q    Median        3Q       Max  
## -1.97810  -0.21406  -0.03454   0.47107   3.04502  
## 
## Coefficients:
##               Estimate Std. Error z value Pr(>|z|)    
## (Intercept) -42.214032   7.617440  -5.542 2.99e-08 ***
## indus        -0.213126   0.073236  -2.910  0.00361 ** 
## nox          80.868029  16.066473   5.033 4.82e-07 ***
## age           0.003397   0.012032   0.282  0.77772    
## dis           0.307145   0.190502   1.612  0.10690    
## rad           0.847236   0.183767   4.610 4.02e-06 ***
## tax          -0.013760   0.004956  -2.777  0.00549 ** 
## ---
## Signif. codes:  0 '***' 0.001 '**' 0.01 '*' 0.05 '.' 0.1 ' ' 1
## 
## (Dispersion parameter for binomial family taken to be 1)
## 
##     Null deviance: 329.37  on 252  degrees of freedom
## Residual deviance: 144.44  on 246  degrees of freedom
## AIC: 158.44
## 
## Number of Fisher Scoring iterations: 8
\end{verbatim}

LDA

\begin{Shaded}
\begin{Highlighting}[]
\NormalTok{Boston.ldafit }\OtherTok{\textless{}{-}}\FunctionTok{lda}\NormalTok{(crime01}\SpecialCharTok{\textasciitilde{}}\NormalTok{ indus}\SpecialCharTok{+}\NormalTok{nox}\SpecialCharTok{+}\NormalTok{age}\SpecialCharTok{+}\NormalTok{dis}\SpecialCharTok{+}\NormalTok{rad}\SpecialCharTok{+}\NormalTok{tax, }\AttributeTok{data=}\NormalTok{Boston.train,}\AttributeTok{family=}\NormalTok{binomial)}
\NormalTok{Bostonlda.pred }\OtherTok{=} \FunctionTok{predict}\NormalTok{(Boston.ldafit, Boston.test)}
\FunctionTok{table}\NormalTok{(Bostonlda.pred}\SpecialCharTok{$}\NormalTok{class, crime01.test)}
\end{Highlighting}
\end{Shaded}

\begin{verbatim}
##    crime01.test
##       0   1
##   0  81  18
##   1   9 145
\end{verbatim}

\begin{Shaded}
\begin{Highlighting}[]
\FunctionTok{mean}\NormalTok{(Bostonlda.pred}\SpecialCharTok{$}\NormalTok{class }\SpecialCharTok{!=}\NormalTok{ crime01.test)}
\end{Highlighting}
\end{Shaded}

\begin{verbatim}
## [1] 0.1067194
\end{verbatim}

K Nearest Neighbors when k =1

\begin{Shaded}
\begin{Highlighting}[]
\NormalTok{train.K}\OtherTok{=}\FunctionTok{cbind}\NormalTok{(indus,nox,age,dis,rad,tax)[train,]}
\NormalTok{test.K}\OtherTok{=}\FunctionTok{cbind}\NormalTok{(indus,nox,age,dis,rad,tax)[test,]}
\NormalTok{Bosknn.pred}\OtherTok{=}\FunctionTok{knn}\NormalTok{(train.K, test.K, crime01.test, }\AttributeTok{k=}\DecValTok{1}\NormalTok{)}
\FunctionTok{table}\NormalTok{(Bosknn.pred,crime01.test)}
\end{Highlighting}
\end{Shaded}

\begin{verbatim}
##            crime01.test
## Bosknn.pred   0   1
##           0  31 155
##           1  59   8
\end{verbatim}

\begin{Shaded}
\begin{Highlighting}[]
\FunctionTok{mean}\NormalTok{(Bosknn.pred }\SpecialCharTok{!=}\NormalTok{crime01.test)}
\end{Highlighting}
\end{Shaded}

\begin{verbatim}
## [1] 0.8458498
\end{verbatim}

when K=100

\begin{Shaded}
\begin{Highlighting}[]
\NormalTok{train.K}\OtherTok{=}\FunctionTok{cbind}\NormalTok{(indus,nox,age,dis,rad,tax)[train,]}
\NormalTok{test.K}\OtherTok{=}\FunctionTok{cbind}\NormalTok{(indus,nox,age,dis,rad,tax)[test,]}
\NormalTok{Bosknn.pred}\OtherTok{=}\FunctionTok{knn}\NormalTok{(train.K, test.K, crime01.test, }\AttributeTok{k=}\DecValTok{100}\NormalTok{)}
\FunctionTok{table}\NormalTok{(Bosknn.pred,crime01.test)}
\end{Highlighting}
\end{Shaded}

\begin{verbatim}
##            crime01.test
## Bosknn.pred   0   1
##           0  21   6
##           1  69 157
\end{verbatim}

\begin{Shaded}
\begin{Highlighting}[]
\FunctionTok{mean}\NormalTok{(Bosknn.pred }\SpecialCharTok{!=}\NormalTok{crime01.test)}
\end{Highlighting}
\end{Shaded}

\begin{verbatim}
## [1] 0.2964427
\end{verbatim}

Conclusion The LDA model states that the nox, age, and media variables
positively correlate with the probability of a high crime rate (above
median). This variable has a negative correlation with the predicted
probability. Using the LDA model, the error rate is slightly lower than
that of the logistic model with 18.5\%. After considering the results of
each classification method, logistic regression had the lowest trial
error rate of 9.09\%. When revisiting the logistic regression model, the
only statistically significant variables were: Indus, nox, rad, and tax.
All modeling methods contain the same variables, so comparisons are most
straightforward; these variables are graphically determined to have the
best association with crime01. K nearest neighbors with k = 1 have the
highest error rate, thus making the model ineffective in classification.
It should also be noted, as the value of K increased, the error rate
improved but was still not as low as logistic regression or LDA.

\end{document}
